
%!TEX encoding = IsoLatin
%!TEX root = ./rapport.tex
\subsection{Appareil de mesure}
\subsubsection{Facteur de proportionnalit� entre puissance et vitesse}
Le but de cet appareil de mesure est de calculer le facteur de proportionnalit� $\lambda$ entre la puissance et la vitesse du conducteur. num�ro de la relation.
En laissant tomber un poids qui est ralenti par la force de Laplace, on observe le poids descendre sur une longueur $L$ dans un temps $\Delta t$. On peut alors d�terminer la vitesse du conducteur car on conna�t le rayon de l'enrouleur $r_{E}$ et le rayon du conducteur $r_{3}'$. 
On admet que sur la distance $L$ le poids n'est pas acc�l�r� et que donc on a une vitesse constante sur cette distance.\\
Reprenons la formule(num�ro de la formule) 
$$\fbox{$ F_{M}=\lambda v$}$$
Dans notre exp�rience on a suppos� la vitesse du poids constant et on peut donc d�duire que la force gravitationnelle sur le poids par la Terre est �gale � la force de Laplace $F_{M}$ \\
$$F_{M} = P$$\\
$P$ �tant la force gravitationnelle du poids en [N]\\
Or $$P = m.g$$\\

$m$ �tant la masse du poids en [kg]\\
La vitesse du poids est la m�me que celle de la bo�te $v_{B}$ car les deux sont reli�s par la m�me corde.\\
On conna�t donc la vitesse angulaire de la tige et donc du conducteur $\omega$\\
$$\omega= \frac{v_{B}}{r_E}$$\\
o� $r_{E}$ est le rayon de la bo�te et donc de l'enrouleur.\\
Et comme le rayon du conducteur est $r_{3}'$ on a la vitesse du conducteur $v$\\

 $$v = \frac{v_{B}r_{3}'}{r_{E}}$$\\
 
 D'ailleurs la vitesse de la bo�te est donn�e par:


$$v_{B} = \frac{L}{\Delta t}$$\\
o� $L$ est la distance parcourue par le poids\\
 $\Delta t$ le temps �coul� pour passer la distance $L$\\

Et donc par(num�roter) on a

$$v = \frac{r_{3}'. L}{\Delta t.r_{E}}$$

Finalement :

$$\lambda = \frac{\Delta t.r_{E}.m.g}{r_{3}'. L}$$

\subsubsection{Calcul de puissance cr�e}