%!TEX encoding = IsoLatin
%!TEX root = ./rapport.tex

\par \textbf{ \og�Ergom�tre de Joule-Foucault�\fg, par Lecomte Marceau, Marchant Nikita, Pletschette Max, Sa�d Salim et Vekemans Benoit, Universit� Libre de Bruxelles, 2011-2012}\\ Nombre de mots~: \num{6478}\\

\par Cette ann�e, le projet BA1 consiste en la construction d'un ergom�tre permettant de chauffer de l'eau  � l'aide des courants de Joule-Foucault. La finalit� est de pouvoir chauffer un volume de \num{10} litres d'eau de \num{3} degr�s en \num{15} minutes, et de d�velopper une puissance de 250 Watt pendant au minimum une minute.

\par Le premier objectif du groupe a �t� de comprendre le principe des courants de Joule-Foucault et de mod�liser leur action sur un conducteur en mouvement.

\par Le dispositif d'entrainement choisi par le groupe est un cyclo-ergom�tre. Celui-ci, d�riv� d'un v�lo conventionnel, est reli� par une cha�ne au caisson contenant le frein magn�tique, partie majeure de l'ergom�tre de Joule-Foucault, constitu� d'une plaque de cuivre et de 8 paires d'aimants dispos�s de part et d'autre de cette plaque.\\ \\ \\


\par\textbf{\og�Joule-Eddy's ergometer\fg, by Marceau Lecomte, Nikita Marchant, Max Pletschette,  Salim Sa�d and Benoit Vekemans, Universit� Libre de Bruxelles, 2011-2012}\\


\par This year's multidisciplinary project of BA1 was to build an ergometer that was able to heat water thanks to the Eddy currents principle. The main goal was to heat a volume of \num{10} liters of water by \num{3} degrees in \num{15} minutes, and to produce a power of \num{250} Watts for at least one minute. The first objective of the group is to understand how Eddy currents work. It has to model it's action on a conducting plate. The chosen machine is a cyclo-ergometer. A regular bicycle was modified for this purpose. The bike is linked to the magnetic brake by an axle and a chain. This system consists of a copper plate rotating between 8 pairs of magnets arranged on either side of the plate.