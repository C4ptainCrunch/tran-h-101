%!TEX encoding = IsoLatin
%!TEX root = ./rapport.tex
\newpage
\section{Conclusion}

Le projet tire � sa fin. L'ergom�tre a �t� construit avec succ�s et il est donc temps pour le groupe d'�crire tout ce qu'il a appris durant ce travail. Ce projet nous a enrichi dans bien des domaines dont quelque-uns sont d�velopp�s dans cette conclusion.

La premi�re chose que ce travail nous a appris, c'est � faire des recherches sur des sujets scientifiques dont on n'a au pr�alable aucune notion. Chercher des livres dans une biblioth�que scientifique, trouver des sources fiables sur internet, etc. 

Ce travail nous a �galement appris que l'�criture d'un rapport scientifique n'est pas une chose qui s'improvise. Elle se fait notamment en parall�le avec le d�roulement du projet.
Ce projet nous a �galement donn� de la joie, particuli�rement de savoir que notre ergom�tre fonctionnait presque parfaitement, apr�s avoir r�ussi � r�soudre tous les probl�mes. 

Tout au long du parcours, nous avons �galement appris que certaines parties d'un projet �taient moins amusantes que d'autres, mais qu'elles ne devaient en aucun cas �tre n�glig�es. Ces t�ches moins dr�les sont notamment aussi indispensables au bon d�roulement d'un projet que les parties agr�ables. 

Tous ces �l�ments et bien d'autres encore ont donc permis � ce groupe d'�voluer, tant du c�t� humain que du c�t� technique de la gestion d'un projet.